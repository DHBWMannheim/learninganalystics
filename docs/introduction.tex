% !TEX root =  master.tex
\chapter{Einleitung}
\section{Motivation und Zielsetzung}
In den letzten Jahren hat Computertechnologie rasant an Leistungsfähigkeit und Funktionsumfang zugenommen.
Längst sind Computer oder Mobilgeräte nicht mehr aus dem Alltag wegzudenken.
Umso wichtiger ist es, diese Technologien sinnvoll in die Tätigkeiten von Menschen zu integrieren und durch diesen einen wertvollen Mehrwert zu generieren.
Immer mehr Themenbereiche sind und müssen digitalisiert werden.

Zum Anlass des aktuellen Weltgeschehens sehen wir die Notwendigkeit, solche Technologien auch in den Hochschul-Kontext einzubringen.
Die aktuell \enquote{Corona-Krise} hat extreme Auswirkungen und macht sich in vielen Bereichen bemerkbar.
Mit einer der Bereiche, die am stärksten betroffen sind, ist das Bildungswesen.
So sind bereits im April 2020 die Schulen von 192 Ländern geschlossen worden.\autocite[S. 845]{Donohue2020}
Die geänderten Anforderungen an Schüler und Studenten durch neue Konzepte wie \enquote{Home-Schooling} erfordern auch Anpassungen im Hochschulkontext.

Mit der von uns entwickelten Anwendung beabsichtigen wir die entstandene Distanz zwischen Studenten und Dozenten zu verringern. 
Mithilfe der Webanwendung möchten wir allen Studierenden und Dozierenden an der DHBW eine Möglichkeit bieten, ihren Hochschulalltag zu organisieren. 
Unsere Vision ist, dass unsere Anwendung einen Mehrwert für Studierende und Dozierende generiert wird.
Daher sollen möglichst alle Aufgaben des Hochschulalltags mit unserer Webanwendung unterstützt werden. 
Im Rahmen dieses Moduls soll ein Grundstein gelegt werden, auf dem weitere Entwicklung aufgebaut werden kann und auch soll.

\clearpage
\section{Aufbau der Arbeit} % TODO: Dieses Kapitel anpassen
In dieser Arbeit werden zunächst die Anforderungen an eine solche Anwendung betrachtet.
Dafür wird als erstes das Thema dieses Projektes abgegrenzt, um den Fokus unseres Vorhabens ausführlich darzustellen.
Anschließend wird der aktuelle Ist-Zustand analysiert, um herauszufinden wie Studenten und Dozenten am besten unterstützt werden können.
Ziel ist es festzustellen, welche funktionalen und welche nicht-funktionalen Anforderungen durch die Anwendung erfüllt werden sollten.

Basierend auf den festgestellten Anforderungen wird ein Konzept ausgearbeitet.
Im Konzept wird nicht nur auf die Funktionalitäten, wie sie Nutzer verwenden können eingegangen, sondern auch auf die technischen Hintergründe.

Anschließend wird die Umsetzung des Konzeptes beschrieben.
Der Fokus liegt dabei auf der Schaffung von Transparenz bezogen auf die verarbeiteten Daten.
Als letztes Hauptkapitel wird die Nutzung der Anwendung durch dritte Personen wie Studenten und Dozenten beschrieben.
% TODO: Hier noch schauen. Soll die Nutztung tatsächlich als eigenes Kapitel geschrieben werden? ist das nicht schon in der Konzeption, und Umsetzung?

Das Nutzerhandbuch stellt eine eindeutige und verständliche Nutzeranleitung dar.
Mithilfe dieser werden alle Funktionalitäten der Endanwendung erklärt.
% TODO: Das auchnochmal reinschauen.   Brauchen wir wirklich ein Kapitel in dem nochmal die ganzen Anforderungen validiert werden? Das steht doch dann auch schon in Konzeption + Umsetzung ?
% TODO: Hier nochmal sätze bauen

Den Abschluss der Dokumentation ist die Zusammenfassung.
Diese besteht aus zwei Teilen.
Das Fazit betrachtet das Projekt als Ganzes und vergleicht die schlussendliche Form mit den in der Einleitung hervorgestellten Zielen. Damit werden alle Tätigkeiten innerhalb dieses Projekts abgerundet abgeschlossen dargestellt.
Anschließend wird ein Ausblick hervorgestellt, an welchen Stellen Folgeprojekte aufbauen können. Was wir aus aktueller Sicht als sinnvoll betrachten, auf was dabei zu achten ist und was vermieden werden sollte.


% TODO: Hier stand noch was drin, dass auch Datenschutz sachen und so beschrieben. Das weiß ich nicht. Das kann man hier reinschreiben, wenn wir das tatsächlich machen. 