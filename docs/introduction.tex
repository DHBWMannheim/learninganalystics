% !TEX root =  master.tex
\chapter{Einleitung}
\section{Motivation und Zielsetzung}

Mithilfe der Studypro-Webanwendung möchten wir allen Studierenden und Dozierenden an der DHBW eine Möglichkeit bieten, ihren Hochschulalltag zu organisieren. 
Unsere Vision ist, dass für Studierende und Dozierende mit dieser Anwendung ein Mehrwert generiert wird. Daher sollen möglichst alle Aufgaben des Hochschulalltags mit unserer Webanwendung unterstützt werden. 
Im Rahmen dieses Moduls soll ein Grundstein gelegt werden, auf dem weitere Entwicklung aufgebaut werden kann und auch soll. 
Daher können wir innerhalb dieser begrenzten Zeit nicht alle Vorhaben umsetzen und weisen am Ende der Dokumentation auf weitere mögliche Vorgehensschritte hin, welche von unseren Nachfolgern angegangen werden können.

Dozierende haben aktuell wenig Möglichkeiten zu erfahren, wie ihre Vorlesung in verschiedenen Kursen angenommen wird, was der Lernstand der Studierenden der einzelnen Kurse ist und wie die Stimmung innerhalb der Kurse bezüglich der anstehenden Prüfungsleistung. 
Aktuell (Dezember '20) finden zum zweiten Mal ausschließlich Online-Vorlesungen statt, sodass diese Möglichkeiten für Dozierende nochmals relevanter sein könnten.


\section{Aufbau der Arbeit} % TODO: Das hier dann nochmal anpassen

Diese Projektdokumentation beginnt mit einer Anforderungsanalyse. In dieser ist der aktuelle Ist-Zustand dargestellt.
Unsere Anforderungen an das Ergebnis dieses Projekts sind darauffolgend genannt. Der letzte Teil der Anforderungsanalyse behandelt die thematische Abgrenzung dieses Projekts, um den Fokus unseres Vorhabens ausführlich darzustellen.

Im nächsten Kapitel sind einige Grundkonzepte er- und bearbeitet, auf welche dieses Projekt aufbaut. Diese sind von theoretischer oder technischer Natur.\\
Die fachlichen Abschnitte befassen sind einerseits rund um die Thematik, wie Dozierenden passende und nützliche Daten zur Verfügung gestellt werden können, ohne datenschutzrelevante und persönliche Informationen von Studierenden Preis zu geben. 
Andererseits wird der Hintergrund von Autorisierung und Authentisierung näher beleuchtet, um eine sichere Registrierung, Anmeldung und Kurseinschreibung von Dozierenden und Studierenden gewährleisten zu können.

Auf technischer Seite wird das Thema "Backend as a Service" betrachtet, welche Vor- und Nachteile das Konzept mit sich bringt und weshalb die Entscheidung gefallen ist, dies in unserem Projekt einzusetzen.\\
Die Datenstrukturen sind besonders relevant, da hieraus die auswertbaren Analysedaten generiert werden sollen. Dabei ist es wichtig, bereits von Anfang an keine Designfehler einzubauen, die als permanente technische Schulden durch das Projekt geschleußt werden.\\
Die Datenschutzthematik ist auf technischer Ebene gesichert, damit Studierenden dies versichert werden kann. 

Im folgenden Konzeptionskapitel wird darauf eingegangen, wie konkret vorgeganen wird. 
Hierbei werden die Frontendarchitektur, Datenmodell und Datenhaltung,  Datensicherheit, Datenschutz und Grundlage für die Datenauswertung genauer erklärt und klar dargestellt, was in welchen Programmabschnitten erreicht werden soll und was dabei zu beachten ist.

Das Implementierungskapitel verdeutlicht, wie die verschiedenen Konzepte innerhalb des Konzeptionskapitels realiesiert werden können. Hier können die Themen Backend, Frontend, Tools und Datenauswertung für Dozenten unterschieden werden.

Das Nutzerhandbuch stellt eine eindeutige und verständliche Nutzeranleitung dar. Mithilfe dieser werden alle Funktionalitäten der Endanwendung erklärt.

Den Abschluss der Dokumentation ist die Zusammenfassung. Diese besteht aus zwei Teilen.
Das Fazit betrachtet das Projekt als Ganzes und vergleicht die schlussendliche Form mit den in der Einleitung hervorgestellten Zielen. Damit werden alle Tätigkeiten innerhalb dieses Projekts abgerundet abgeschlossen dargestellt.
Anschließend wird ein Ausblick hervorgestellt, an welchen Stellen Folgeprojekte aufbauen können. Was wir aus aktueller Sicht als sinnvoll betrachten, auf was dabei zu achten ist und was vermieden werden sollte.

