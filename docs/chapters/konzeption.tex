% !TEX root =  ../master.tex
\chapter{Konzeption}
Bevor mit der Implementierung gestartet werden kann wird ein Konzept erstellt, welches als Leitfaden für die Implementierung dienen soll.
Ziel ist ein konkreter Entwurf, welcher die technischen und grundlegenden Funktionalitäten beschreibt.
Dadurch soll verhindert werden, dass die aufwendige Implementierung fehlschlägt oder mehrfach Änderungen durchgeführt werden müssen, welche den Implementierungsaufwand vergrößern würden.


















\section{Konzeption der Anwendungsarchitektur}
Die Grundlage jeder Anwendung ist der Aufbau und die Struktur der Infrastruktur, auf welcher die Anwendung später ausgeführt werden soll.
Aus diesem Grund wird diese als Erstes betrachtet.

% -TODO: Thin-Client --> Hier eingearbeitet

Webanwendungen werden üblicherweise in einem Client-Server-Modell entwickelt. Im Konkreten kann hier von einer Thin-Client Architektur gesprochen werden, da jegliche Anwendungsdatenverarbeitung Server-seitig durchgeführt wird \autocite{definitionOfThinClient}. % TODO: \autocite{Leff} --> ?
Der Client übernimmt dabei sämtliche Funktionalitäten, die das Nutzerinterface betreffen. Über diese Aufteilung von Verantwortlichkeiten können Wartung und Sicherheit durch die zentrale serverseitige Verwaltung in einem Rechenzentrum einfacher gesichert werden. Falls der Client selbst Daten verarbeiten und speichern würde, könnte dies zu inkonsistenten Daten und Sicherheitslücken führen, da unterschiedliche Datenbestände zwischen Client und Server entstehen könnten. 
Visualisierung und die Interaktion mit dem Nutzer fallen hierbei in die Aspekte des Nutzerinterfaces, worüber diesem die Interaktion mit den zentralen Datenbeständen ermöglicht wird. Dabei wird jedoch keine Anwendungslogik ausgeführt \autocite{thinClientArchitectureOverview}. Es ist allerdings wichtig zu betonen, dass diese Nutzerschnittstelle essentiell für den Erfolg der Anwendung ist. Eine maximale User Experience \autocite{definitionUserExperience} zu bieten ist ein kritischer Faktor für die Qualität der Nutzerschnittstelle \autocite{thinClientArchitectureOverview}.\\
Entsprechend wird die gesamte Geschäftslogik, Datenhaltung und der Datenaustausch von einem zentralen Server verwaltet. 
In \autoref{fig:clientServerAufbau} ist eine solche Architektur vereinfacht dargestellt, wobei zwischen dem Anwendungs- und dem Datenbankserver unterschieden ist. Der Anwendungsserver beinhaltet hierbei sämtliche Geschäftslogik- und Datenaustauschzuständigkeiten. Der Datenbankserver ist hingegen ausschließlich für die persistente Datenhaltung zuständig.

\begin{figure}[h]
    \centering
    \includegraphics[width=.9\textwidth]{img/ClientServer.png}
    \caption{Vereinfachter Client-Server-Aufbau}
    \label{fig:clientServerAufbau}
\end{figure}

Praktisch dargestellt, ruft der Nutzer mithilfe seines Browsers die Webseite von einem Server ab.
Der Browser verarbeitet mithilfe der implementierten Geschäftslogik die Anfrage des Nutzers und greift auf die Daten des Datenbankservers zu. Diese Daten werden schließlich dem Nutzer zur Verfügung gestellt, welche ihm in der Nutzerschnittstelle angezeigt werden und für ihn an  dieser Stelle aufbereitet einsehbar sind.\\
Dabei greift der Nutzer über ein wohldefiniertes \ac{API} auf den Anwendungsserver zu, um diesen dargestellten Prozess abzusichern.

%Dieser läd die notwendigen Informationen auf ähnliche Weise von einem Datenbankserver und bereitet diese gegebenenfalls auf. edit pdm: im Absatz obendrüber bearbeitet
%Der Nachteil von Client-Server-Modellen ist der hohe Aufwand im Bereich der Skalierung. edit pdm: im Absatz untendrunter bearbeitet

Nach den Anwendungsanforderungen wird eine hohe Leistungsfähigkeit benötigt, die auch mit Leistungsspitzen klar kommen muss. Da anzunehmen ist, dass der Großteil der Nutzer in der Mitteleuropäischen Zeitzone vorzufinden sein wird, können solche Leistungsspitzen morgen zu Vorlesungsbeginn, nach der Mittagspause oder am frühen Abend auftreten. Bei diesen Leistungsspitzen ist eine hohe Serverkapazität notwendig.
Hierbei müssen sowohl die implementierte Serverlogik, als auch die zugrundeliegende Serverinfrastruktur konsistente Daten mit akzeptablen Antwortzeiten liefern. Diese Vorgänge fallen in den Bereich der Skalierung, welcher sowohl für die Serverlogik, als auch für Infrastruktur einen hohen Aufwand birgt.\\
Um dieses Problem zu lösen gibt es zwei Lösungsansätze. In der ersten Lösung können permanent Serverressourcen bereitgestellt werden, die auch bei Leistungsspitzen nicht überlastet sind und ausfallen, jedoch würden bei dieser Lösung viele Ressourcen in \enquote{Ruhephasen} ungenutzt bleiben, was unökonomisch und kostspielig wäre.\\
In der zweiten Lösung können notwendige Serverressourcen an Leistungsspitzen dynamisch hinzugezogen werden. Hierbei besteht jedoch die Schwierigkeit, diese Leistungsspitzen rechtzeitig zu erkennen, um frühzeitig Serverressourcen steigern zu können. Sofern diese Schwierigkeiten bei dem zweiten Lösungsansatz umgesetzt werden können, ist dies eine optimale Lösung, jedoch ohne weiteres nicht einfach umzusetzen.\\

% FIXME: Das doppelt sich mit den Grundlagen
Die Lösung dieses Problems hat sich vor Jahren unter dem Begriff des Cloud Computings \autocite{definitionOfCloudComputing} etabliert. Hierbei können je nach Nutzeranforderungen Ressourcen bereitgestellt und auch wieder entzogen werden. Der Entwickler einer Anwendung muss nur die verwendeten Ressourcen (somit Netzwerkverkehr, Prozessorrechenzeit und generelle Verarbeitungszeit) bezahlen. Ressourcen können hiermit dynamisch bei Leistungsspitzen in großem Umfang zur Verfügung stehen.% Das heißt, sobald eine Leistungsspitze entsteht müssen erst aufwändig zusätzliche Server hochgefahren werden, welches die Anwendung kurzzeitig für viele Nutzer unnutzbar macht. edit pdm: Im Absatz obendrüber behandelt

Des Weiteren müssen diesen Ressourcen nicht in Form von eigenen, abgeschlossenen Servern zur Verfügung gestellt werden, sondern die Geschäftslogik von Entwicklern kann in einer Serverless Computing Umgebung \autocite{definitionOfServerlessComputing} ausgeführt werden. Hierbei wird lediglich die Anwendungslogik in der Cloud (zum Beispiel in Form einer Funktion) ausgeführt, ohne dabei auf einem dezidierten Server zu laufen.

Dieses Serverless Commputing kann in \ac{BaaS} und \ac{FaaS} unterteilt werden \autocite{whatIsServerless}. Hierbei ist zu unterscheiden, dass bei \ac{BaaS} dem Entwickler weitere Backend-Funktionalitäten zur Verfügung gestellt werden, welche er in seiner Anwendungs- (und damit Geschäfts-)logik verwenden kann, womit die Skalierbarkeit der entwickelten Lösung nochmals an vielen Stellen verbessert werden kann. Aus diesem Grund haben wir uns für einen solchen \enquote{Serverless}-Ansatz entschieden.

In \autoref{sub:firebase}

\begin{figure}[h]
    \centering
    \includegraphics[width=.9\textwidth]{img/Firebase.png}
    \caption{Vereinfachter Firebase-Aufbau}
    \label{fig:firebaseAufbau}
\end{figure}














\section{Konzeption der Funktionalitäten}\label{sec:konzeptionFunktionalitaeten}


\subsection{Kurse}

Informationen sollen in Kurse unterteilt werden.
Aus diesem Grund wird eine Möglichkeit benötigt, wie Nutzer Kurse wechseln können und erkennen können zu welchem Kurs die Informationen gehören.
Auch hier gibt es mehrere Möglichkeiten, wie eine solche \ac{UX} gestaltet werden kann, die jeweils verschiedenen Vor- und Nachteilen besitzen.

Eine Möglichkeit ist die gesamte Anwendung innerhalb eines Kurses anzubieten.
Das heißt, der Nutzer hat nach dem Login die Möglichkeit einen seiner Kurse auszuwählen.
Dieser Kurs wird anschließend geladen und der Nutzer sieht nur die Informationen dieses Kurses.
Möchte man den Kurs wechseln, gibt es ein Bento-Menü, über welches ein anderer Kurs ausgewählt werden kann.
Eine solche \ac{UX} hat den Vorteil, dass sich ein Nutzer vollständig auf einen Kurs konzentrieren kann und so nicht Informationen zwischen verschiedenen Kursen durcheinander wirft.
Gleichzeitig kann dies aber dazu führen, dass ein Nutzer eventuell den Überblick über seine Kurse verliert und so einen Kurs womöglich vernachlässigt.
Auch Kursübergreifende Informationen, z.\,B. TODOs, können nur schwer abgebildet werden, ohne das die Nutzung der Anwendung erklärungsbedürftig wird.

Als zweite Möglichkeit kommt eine virtuelle Trennung zum Einsatz.
Dabei sieht der Nutzer alle Informationen und eine Trennung dieser findet nur über die Navigation statt.
So besitzt die Anwendung für jeden Kurs einen eigene Sektion, in der die Informationen dargestellt werden.
Dadurch kann der Nutzer stets alle Kurse sehen und die \ac{UX} wird vereinfacht.



\subsection{Rollenzuweisung}

Aus den Anforderungen geht hervor, dass es zwei Nutzerrollen gibt: Dozenten und Studenten.
Dozenten sollen in der Anwendung ihre Kurse verwalten können.
Ein Kurs orientiert sich dabei an der Organisation, die in der Realität vorliegen.
Auch in vorhandenen Tools wie beispielsweise \enquote{Moodle} sind Studenten anhand von Kursen organisiert.
Aufgrund der Vielzahl an solchen Lösungen scheint sich eine solche Organisation zu bewähren.
Aus diesem Grund liegt es nahe, das auch die zu entwickelnde Anwendung eine solche Struktur nutzen sollte.
% TODO: Wie Werden Kurse erstellt

Nachdem nun geklärt wurde, wie die Organisation von Studenten und Dozenten stattfindet, muss nun die Zuweisung von Studenten zu Kursen die Zuweisung von Dozenten konzipiert werden.
Zunächst wird die Studentenzuweisung betrachtet.
Für die Zuweisung von Studenten kommen mehrere Ansätze in betracht, die sich in der Person unterscheiden, die die Zuweisung durchführt:
\begin{enumerate}
    \item Zuweisung durch Anwendungsadministrator
    \item Zuweisung durch Dozenten
    \item Zuweisung durch Studenten
\end{enumerate}
Als erster Ansatz kommt die Zuweisung durch einen Anwendungsadministrator in betracht.
Ein solcher Administrator ist ein zentraler Mitarbeiter der DHBW, welcher komplette Autorität über die Anwendung besitzt.
Von einem solchen Ansatz wird abgesehen, da der Verwaltungsaufwand sehr hoch ausfällt.
Für jeden Kurs über 20 Studenten zuzuweisen, und das für mehrere Kurse pro Semester und Vorlesung, scheint nicht realistisch.

Der zweite ist der Ansatz der Zuweisung durch einen Dozenten.
In der aktuellen DHBW-Organisation besitzen Dozenten bereits eine Liste an Studenten für ihre Vorlesung.
Diese Liste dient zur Anwesendheitskontrolle.
Demnach können Dozenten diese Liste für die Zuweisung in der Anwendung nutzen.
Der Nachteil eines solchen Ansatzes ist es aber, dass Studenten ihre Daten in der Anwendung hinterlassen müssen und diese durch andere Dozenten gegebenenfalls einsehbar wären.
Zusätzlich geht die Anonymität, welche durch Matrikelnummern gegeben ist eventuell verloren.
Insgesamt ist dieser Ansatz aus Datenschutzaspekten besonders kritisch. % TODO: Beim Admin nicht so, weil dhbw

Als Letzter Ansatz ist die Zuweisung durch den Studenten.
Denkbar sind hier mehrere Ansätze: Eine Liste aus Kursen oder über einen Kursidentifikator.
Eine Liste mit Kursen vereinfacht die Zuweisung durch den Studenten, da Kurse schnell entdeckt werden können und eventuell zusätzliches Wissen vermittelt werden kann, wenn der Student sich für mehrere ähnliche Kurse einträgt.
Nachteil ist aber die unmittelbare Verwaltung und Übersicht über die Kurse.
Eine Zuweisung über einen Kursidentifikator (kurz: Schlüssel) hat den Nachteil, dass diese Schlüssel erst aufwändig über einen weiteren Kommunikationsweg (z.\,B. Email oder direkt in einer Vorlesung) mitgeteilt werden muss.
Dafür besitzen Studenten aber nur Zugriff auf die für sie relevanten Vorlesungen.
In \enquote{Moodle} ist die Zuweisung zu Kursen auf freiwilliger Basis anhand von Einschreibeschlüsseln gelöst.
Aus diesem Grund ist dieser \enquote{Workflow} bereits für Studenten bekannt und eine Anpassung an die neue Anwendung ist schnell möglich.
In der Tat nutzen viele existierende Anwendungen ein solches System.
Beispiele hierfür sind Zoom oder Google Meets.


\subsection{Registrierung}
Nutzer der Anwendung können sich selbst registrieren.
Daraus ergibt sich die eine Rollen Problematik: Woran kann die Anwendung erkennen, welcher Nutzer ein Student ist und welcher Nutzer ein Dozent ist.
Sofern Nutzer dies selber angeben können, bräuchte es gegebenenfalls eine Validierung durch die DHBW, wodurch erneut die oben genannten Nachteile einer Zuweisung durch einen Administrator greifen.
Ein Ansatz, bei dem Nutzer dies selber verwalten können wird auch hier als besser angesehen.
Aus diesem Grund wird der folgende Ansatz verwendet:
Statt festdefinierte Rollen (Student und Dozent) zu besitzen, kann jeder Student selbst sowohl Student, als auch Dozent sein.

Jeder Nutzer ist zunächst keiner Rolle zugeordnet.
Jeder Nutzer kann einen Kurs erstellen, wodurch dieser Nutzer automatisch zu einem Dozent für diesen Kurse.
Sobald sich ein Nutzer mithilfe des Einschreibeschlüssels für einen Kurs anmeldet wird der Nutzer automatisch für diesen Kurs zu einem Studenten.
Dadurch können auch zuvor nicht vorgesehene Nutzerbeziehungen entstehen.
Beispielsweise kann ein Dozent sich in einen weiteren Kurs einschreiben, falls er sich in einer weiteren Richtung weiterbilden möchte.
Oder die DHBW kann sich in für einen Kurs einschreiben um die Qualität einer Vorlesung zu validieren.


Ein Vorteil eines solchen Ansatzes ist es, dass auch Studenten Kurse erstellen können und so Lerngruppen gefördert werden.
Denkbar ist beispielsweise ein Student, welcher Nachhilfe anbietet.
Dadurch wird die Nutzung der Anwendung gefördert.








\subsection{TODOs}
//TODO:

\subsection{Karteikarten}
//TODO: Einleitung zu Anforderungen
Für das Lernen mit den Karteikarten gibt es eine Vielzahl an unterschiedlichen Algorithmen. Unsere Anwendung implementiert einen Algorithmus aus dem Bereich der \enquote{Spaced Repetition Systems}. In die deutsche Sprache übersetzt heißt das so viel wie "Wiederholen ohne Lücken". Das System hinter diesen Algorithmen besteht darin, die entsprechenden Informationen genau dann zu wiederholen, wenn das menschliche Gehirn sie fast schon vergessen hätte.\autocite[Vgl.][]{Tabibian3988} Die einzelnen Karteikarten werden nacheinander abgefragt und bei richtiger Antwort in zunehmenden Zeitintervallen immer wieder überprüft. Durch diesen Abstandseffekt soll gezielt das Langzeitgedächnis trainiert werden, damit die Inhalte auch über eine Prüfung hinaus im Gedächtnis behalten werden. Die unterschiedlichen Algorithmen dieser Klasse unterscheiden sich lediglich in der Wahl der Zeitabstände. 
Diese Anwendung implementiert das Super-Memo System des polnischen Neurobiologen Piotr Wozniak. Es definiert folgende Zeitabstände:
\begin{itemize}
	\item 20 Minuten
	\item 24 Stunden
	\item 48 Stunden
	\item 10 Tagen
	\item 30 Tagen
	\item 60 Tagen
\end{itemize}  
Kann eine Frage nicht beantwortet werden, so wird diese direkt wiederholt und durchläuft die definierten Zeitintervalle von vorn. \autocite[Vgl.][]{BaileyuDavey}



\subsection{Informations}
//TODO

\subsection{Dashboard}
//TODO

\subsection{Auswertungen}
//TODO




























\section{Datenstruktur}
Aus den in \autoref{sec:konzeptionFunktionalitaeten} entworfenen Funktionalitäten lässt sich ein Entwurf für die Datenhaltung erstellen.

Die Datenspeicherung findet vollständig innerhalb von Firebase statt.
Der hierfür genutzte Service heißt \enquote{Firestore}.
Dabei werden Daten nicht wie in einer klassischen relationalen Datenbank (\ac{RDB}) in Tabellen mit Spalten und Zeilen gespeichert, sondern in Collections aus Dokumenten. Dies erlaubt eine größere Flexibilität in der Entwicklung.


In \autoref{fig:erDiagramm} ist das zugrundeliegende Datenmodell abgebildet, welches in der Implementierung verwendet wird.

\begin{figure}[ht!] % TODO: Hier fehlen die streaks 
    \begin{center}
        \includegraphics[width=\textwidth]{img/Integrationsseminar ER.png}
        \caption{ER-Diagramm}
        \label{fig:erDiagramm}
    \end{center}
\end{figure}





