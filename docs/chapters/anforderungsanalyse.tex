% !TEX root =  ../master.tex
\chapter{Anforderungsanalyse}
\section{Funktionale Anforderungen}


\section{Nicht-Funktionale Anforderungen}


\section{Ist-Analyse}
Grundlage dieser Arbeit ist die Vision der DHBW, Learning Analitics weiter voranzutreiben.
Aus Datenschutzrechtlichen Gründen, wird aktuell die Learning Analytics Erweiterungen und Möglichkeiten der Kursmanagement und Lernplattform Moodle nicht genutzt.
Aus einem bereits umgesetzten Projekt ist ein Prototyp entstanden, der es ermöglichen soll Daten im Verlauf der Vorlesungen mit der Hilfe von Umfragen zu generieren.
Weitere Infos über das Projekt und deren Ergebnisse können hier (https://github.com/dhbw-myla/myla) nachgelesen werden. 
Die dabei entsandene Webanwendung ermöglicht es Umfragen zu erstellen, zu verwalten und auszuwerten.
Für Dozenten gibt es die Möglichkeit, Umfragen zu erstellen und mit ihren Studierenden zu teilen.
Diese haben dadurch die Chance, ihre Meinung zur Vorlesung mitzuteilen, Gelerntes zu verinnerlichen oder auch um Anmerkungen zu gestellten Fragen zu geben.
Die Plattform ermöglicht es dabei, Umfragen als Vorlage zu speichern, sodass diese in unterschiedlichen Kursen verwendet werden können und zeigt die Ergebnisse der Umfragen graphisch an.
Allerdings ist anzumerken, dass wir in unserem Studium neben der Vorstellung der Anwendung für unser Projekt nie mit der Anwendung gearbeitet hatten und diese nach aktuellem Wissensstand auch nicht von der DHBW gehostet wird, was eine breite Nutzung eher unwahrscheinlich macht. 
\section{Anforderungen}
Problematisch bei der Meinungserhebung mittels Umfragen ist häufig die Art und Formulierung der Fragestellung, das Nichtbeantwortungsproblem und ggf. mangelnde Repräsentativität.
Mehr Hintergründe und Lösungsansätze zu diesen und weiteren Problemen lassen sich in der Einschlägigen Fachliteratur nachlesen.
Ziel dieser Arbeit ist es diesen Probleme entgegen zu Wirken, in dem wir eine Anwendung entwickeln, die auch einen direkten Mehrwert für die Studenten hat.
Zwar erbringt die Datenerhebung und Auswertung mittels Umfragen langfristige Mehrwerte, bei der Teilnahme an der Umfrage entsteht jedoch erst einmal ein Aufwand, dessen Nutze ungewiss ist.
Aus diesem Grund ist das Motto "Von Studenten für Studenten" entstanden.
Wir als Studenten der DHBW können sehr gut einschätzen was uns das Studium erleichtern würde und wie wir uns in Bezug auf Klausuren und Prüfungen organisieren.
Mit einer Prozessierung und Automatisierung können Synergien genutzt werden und auch andere Kommilitonen von den für diese Veranstaltungen eingebrachten Aufwänden profitieren. 

Ziel dieser Arbeit ist es eine App zu entwickeln, die Plattform unabhängig als Web-Applikation oder im Browser genutzt werden kann.
Dabei sollen sowohl Studenten als auch Dozenten einen Zugang bekommen.
Um eine möglichst hohe Nutzerzahl erreichen zu können, setzen wir auf eine freiwillige Teilnahme.
Die Anwendungen soll durch ihre Funktionen und die intuitive Bedienbarkeit überzeugen und nicht durch einen Nutzungszwang.
Zweiteres würde unserer Einschätzung nach nur zu einer halbherzigen Nutzung führen, was in einer schlechteren Datenqualität enden würde und damit Auswertungen und deren Interpretation erschweren würde. 

Bei der Erstanmeldung sollen bereits erste Daten über den Lerntyp der Studenten gestellt und gespeichert werden.
Besitzt ein Nutzer einen Account, soll er Kurse anlegen können, zu denen er Informationen wie Datum von Prüfungen erfassen kann.
Für jeden Kurs besteht die TODOs mit Zeildatum zur Strukturierung der eigenen Arbeit anzulegen.
Außerdem können für eine bessere Prüfungsvorbereitungen Karteikarten angelegt werden.
Die Karteikarten können auch in der App gelernt werden.
Dabei soll auf wissenschaftlich fundierte Algorithmen zurückgegriffen werden, die das Lernen erleichtern sollen und dafür sorgen, dass sich das gelernte auch im Langzeitgedächtnis festigt.
Dies ist aus eigenen Erfahrungen in den meist Stressigen Klausurenphasen leider nicht der Fall.
Außerdem sollen die eingetragenen TODOs graphisch angezeigt werden, sodass eine Art Kalender entsteht und Kapazitätsengpässe rechtzeitig erkannt und durch Umplanen verhindert werden können.
Um zu verhindern, dass diese organisatorische Aufwand von jedem Studenten einzeln durchgeführt werden müssen soll es zusätzlich die Funktion geben, dass Dozenten einen Kurs anlegen können.
Diesen können Sie dann mit der Hilfe eines Einschreibeschlüssels unter ihren Studierenden teilen.
Den Kursen können initial bereits Prüfungstermine, TODOs und Karteikarten mitgegeben werden.
Somit entfällt die initiale Erstellung.
Die zu einem Kurs gehörenden Informationen können von den Studierenden weiterhin nach belieben angepasst und ergänzt werden ohne dabei für die anderen eingeschriebenen Kursteilnehmer Daten zu verändern.
Das Erstellen eines Kurses hat für den Dozenten den Vorteil, dass für diesen Kurs Daten zur Verfügung gestellt bekommt, die er zur Anpassung oder Evaluation seiner Vorlesung nutzen kann.
Beispiele hierfür können sein Lerntypen der Studenten, Fortschritt beim Lernen der Karteikarten, etc.
Wichtig ist dabei anzumerken, dass diese Daten vollständig anonym ohne Verweise auf Personen oder andere Identifikationsmerkmale zur Verfügung gestellt werden, sodass auch keine Probleme mit dem Datenschutz vorliegen. 

%TODO: Tabelle mit Anforderungen und kategorisieren nach functional und non-functional Requirenments  
\section{Abgrenzung}
Diese Arbeit wird sich nur mit der ersten Ebene, der Individual Ebene befassen. 

\includegraphics[width=\linewidth,keepaspectratio]{img/Ebenen}
%TODO: Verweiß auf pptx Fr. Honal

Um eine Fortführung und Integration der anderen Ebenen in Zukunft zu ermöglichen, soll die Datenhaltung und Architektur der Anwendung so konzipiert werden, dass eine Erweiterung leicht möglich ist.
Durch die Schnittelle bzw. das zur Verfügung stellen der anonymen Lerndaten wird in diesem Projekt schon die Verbindung zur nächsten Ebene, der Kursebene geschlagen.
Vordefinierte Auswertungen, Berechnungen wird es allerdings aufgrund der gewählten Abgrenzungen erst einmal nicht geben.
Die vorhandenen Daten sollen im Ersten Schritt in einem auswertungstauglichen Format zur Verfügung gestellt werden.
Die Auswertung selbst muss durch den Dozenten selbst erfolgen.
Für die weiteren Ebene sehen wir jedoch die Möglichkeit in weiteren Projekten Ideen umzusetzen, die auf diesem Projekt aufbauen.
Mit diesem Hintergedanken wird in den folgenden Abschnitten die Grundlagen und Konzeption geklärt bzw. erstellt.
