% !TEX root =  ../master.tex
\chapter{Implementierung}
% \section{Backend}
% \subsection{Firebase}
% \subsection{Datenhaltung}



% \section{Frontend}
% \section{Tools}
% \section{Datenauswertung für Dozenten}



\section{Allgemeine Implementierungsdetails}

Bei der Entwicklung der Anwendung haben wir großen Wert auf Datensicherheit und der Einhaltung der DSGVO gelegt.
Ein wesentlicher Bestandteil der unternommenen Maßnahmen ist die Anonymität innerhalb der Anwendung.
Die Anwendung speichert keinerlei Informationen, worüber Nutzer identifiziert werden können.
Bei einer Registrierung muss lediglich eine Email angegeben werden.

\section{Lazy Loading} % Teil von NF1



\section{Responsive Design} % NF3
Nach den nicht-funktionalen Anforderungen ist ein responsive Design pflicht.  
Bei der Entwicklung wurde auf die responsive Darstellung geachtet.
So verhalten sich alle Funktionalitäten einer Anwendung wie es von einer nativen Anwendung erwartet wird.
Dies macht sich besonders in der optimierten Darstellung bemerkbar, bei der der Nutzer nicht zu scrollen oder zoomen verpflichtet ist bemerkbar.
Aber auch andere Funktionalitäten, wie das Wischen der Karteikarten funktioniert auf Geräten mit einem Touchscreen genauso, wie auf Geräten, welche eine Maus verwenden.


% TODO: Vielleicht eher UX schreiben und dann alles sagen, wie z.B. Kontrast, ...
