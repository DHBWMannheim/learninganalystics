% !TEX root =  ../master.tex
\chapter{Prozesse}
Im folgenden werden die durch unsere Anwendung implementieren Prozesse grob als ereignisgesteuerte Prozesskette (EPK) aufgezeigt. Dabei wurden die folgenden Hauptfunktionen identifiziert:
\begin{enumerate}
	\item Login
	\item Auswahl des Farbschematas
	\item Auswahl der Sprache
	\item Suche
	\item Kursverwaltung
	\item Kursfunktionen
	\item Aufgaben
\end{enumerate}

\section{Login}
	\includegraphics[width=\linewidth, keepaspectratio]{img/Prozesse/login}
\section{Auswahl des Farbschematas}
	\includegraphics[width=\linewidth, keepaspectratio]{img/Prozesse/color}
\section{Auswahl der Sprache}
	\includegraphics[width=\linewidth, keepaspectratio]{img/Prozesse/language}
\section{Suche}
	\includegraphics[width=\linewidth, keepaspectratio]{img/Prozesse/search}
\section{Kursverwaltung}
	\begin{center}
		\includegraphics[width=\linewidth, keepaspectratio, angle= 90]{img/Prozesse/coursemanagement}
	\end{center}
\section{Kursfunktionen}
	\begin{center}
		\includegraphics[width=\linewidth, keepaspectratio, angle= 90]{img/Prozesse/coursefunction}
	\end{center}
\section{Aufgaben}
	\begin{center}
		\includegraphics[width=\linewidth, keepaspectratio, angle= 90]{img/Prozesse/tasks}
	\end{center}