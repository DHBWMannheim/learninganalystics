% !TEX root =  ../master.tex
\chapter{Grundlage}
\section{Darlegung des Learning Analytics Konzepts}

%Begriff: Learning Analytics System 
Ein \ac{LMS} hat die grundlegenden Aufgaben, Lernaktivitäten zu verwalten, aufzuzeichnen und über diese zu berichten. Diese \ac{LMS} können aus unterschiedlichsten Komponenten bestehen, die sich mit verschiedenen Aspekten dieser Aufgaben befassen \autocite[S.1]{learningManagementSystemsFieldGuide}. Zusätzlich sind entsprechende Nutzerdaten von Studierenden und Dozierenden zusammen mit den dazugehörigen Lernmaterialien innerhalb des \ac{LMS} zu verwalten. \autocite{learningManagementSystemDefinition}\\
In einer Arbeit von 2013 \autocite[S.253]{SCHOONENBOOM2014247} sind verschiedene Formen von potentiellen Lernaktivitäten bei der Verwendung eines \ac{LMS} vorgestellt und im Rahmen einer Befragung von Studenten und Dozenten ausgewertet worden. Als entscheidende Einflüsse für Lernaktivitäten sind hierbei \enquote{Präsentationen}, \enquote{Referenzen auf weiterführende Inhalte}, \enquote{Fragemöglichkeiten für Studierende}, \enquote{hilfreiche Youtube-Inhalte} und \enquote{Feedbackmöglichkeiten von Dozenten} erkannt worden.

In der selben Arbeit \autocite[S.247]{SCHOONENBOOM2014247} wird auf die drei verschiedenen Ebenen von \ac{LMS} eingegangen, wie Lernaktivitäten dargestellt sind und wie deren Kontext dargestellt ist.
\begin{enumerate}
	\item Auf der ersten Ebene befinden sich Werkzeuge (eng. \enquote{Tools}). Diese stellen konkrete Handlungen dar. Zum Beispiel könnte das Halten von Online-Vorlesungen ein Werkzeug darstellen.
	\item Auf der zweiten Ebene kann mit einem Werkzeug eine Lehr-/Unterrichtsaufgabe (eng. \enquote{instructional task}) unterstützt, beziehungsweise realisiert werden. Beispielsweise können gehaltene Online-Vorlesungen die Unterrichtsaufgabe realisieren, dass Studierenden Lehrinhalte erklärend vorgestellt werden.
	\item Die dritte Ebene stellt letztendlich konkrete Lernaktivitäten dar. Diese können durch eine Kombination von einer oder mehreren Lehr-/ Unterrichtsaufgaben erreicht werden. Um das Beispiel abzuschließen können in einer Online-Vorlesung (Werkzeug) vorgestellte Lerninhalte (Unterrichtsaufgabe) ein Lernaktivität induzieren, nämlich, dass aufmerksame Studierende die behandelten Themen der gehaltenen Vorlesung verstehen.	
\end{enumerate}

Aus all diesen Aktivitäten rund um \ac{LMS}, und im speziellen um Lernaktivitäten, lassen sich Daten erheben, die das Verhalten von Lehrenden und Lernenden darstellbar machen. Generell gilt für die gesamte Interaktion mit \ac{LMS}, dass aus expliziten und impliziten Handlungen Daten und Informationen gewonnen werden können \autocite[S.26]{2012HorizonReport}. Die systematische Analyse dieser Daten wird im Rahmen von \enquote{Learning Analytics} durchgeführt. Die in einem \ac{LMS} verfügbaren Daten können wiederum in verschiedenen Ebenen analysiert werden \autocite[S.2f]{learningAnalyticsImHochschulkontext} \autocite{penetratingTheFogAnalyticsInLearningAndEducation}:

\begin{itemize}
	\item Individualebene: 
	
		In dieser Ebene können Schlüsse zu dem Verhalten von individuellen Studierenden gebildet werden. Hieraus können Lernalternativen für Studierende formuliert werden.
		
	\item Kursebene:
	
		Dozierenden nützliche Informationen zu ihren Kursen zur Verfügung zu stellen geschieht auf dieser Ebene. Sie können für individuelle Handlungsalternativen bei unterschiedlichen Kursen herangezogen werden. 
	
	\item Institutionsebene:
	
		Mithilfe von Daten dieser Ebene kann eine Institution gezielt Ressourcen zuweisen, um den Studienbetrieb bestmöglich zu unterstützen.  
	
	\item Politische Ebene:
	
		Die politische Ebene stellt die abstrakteste Ebene dar. Auf dieser Ebene können fundamentale Eigenschaften des zugrundeliegenden Bildungssystems erhoben und verglichen werden.
	
\end{itemize}

Mithilfe von diesen unterschiedlichen Datenbezugsebenen kann die datengesteuerte Entscheidungsfindung unterstützt werden \autocite[S.19]{theEvolutionOfBigDataAndLearningAnalyticsInAmericanHigherEducation}, so dass von allen unterstützten Akteuren fundiertere Entscheidungen begründet getroffen werden können. Konkret können Bildungsangebote auf individuelle Bedürfnisse und Fähigkeiten abstimmen werden \autocite[S.26]{2012HorizonReport}.



% TODO????? (Irgendwas, um von Learning activities auf Learning analytics zu kommen)



\section{Thematische Einordnung dieser Arbeit}

%Learning management System (parallel zu Moodle)

Im Rahmen dieser Arbeit wird ein \ac{LMS} erarbeitet, dass parallel zu dem bereits existierenden \enquote{Moodle}-\ac{LMS} konzipiert ist. Hierbei wurde darauf geachtet, dass diese Kombination beider eigenständigen \ac{LMS} eine effektive Symbiose darstellt. Lernaktivitäten sollen transparent einem \ac{LMS} zugeordnet werden können und Überschneidungen sind zu minimieren. Gewisse Überschneidungen sind (an vielen Stellen anhand des prototypischen Charakters dieser Erarbeitung) technisch bedingt. Zum Beispiel sind konkrete Kurse in beiden \ac{LMS} parallel zu führen. In einer weiteren Ausbaustufe können entsprechende Schnittstellen zwischen beiden Systemen technisch bedingte Überschneidungen minimieren. Fachliche Themen der \enquote{Learning Analytics}-Komponenten des \ac{LMS} sind folgend erarbeitet, wobei Überschneidungen zu Moodle weitestgehend vermieden wurden. 

Die Erhebung von Daten in der Individual- und Kursebene ist angestrebt. Daten auf Kursebene werden im Kurskontext erhoben (zum Beispiel fallen in diese Kategorie Exmatrikulationsquoten, Abwesenheitsquoten, Aggregierung von Daten der Individualebene, $\ldots$), wohingegen Daten der Individualebene an unterschiedlichen Studierenden erhoben werden (zum Beispiel aus Befragungen oder Interaktionen mit einem \ac{LMS}).

Die Datenschutzthematik mit personenbezogenen Daten nach Art.4 Nr.1 DS-GVO ist durch das Vorhandensein einer eindeutigen Nutzerkennung gegeben. Durch die Einhaltung der Grundsätze der Datenverarbeitung nach Art.5 DS-GVO und des Vorliegen des Erlaubnistatbestands nach Art.6 Abs.1 S.1 lit.f ist die Verarbeitung dieser personenbezogenen Daten aber  zulässig. Hier gilt es jedoch anzumerken, dass der Erlaubnistatbestand an die \enquote{Verarbeitung zur Wahrung berechtigter Interessen} gebunden ist. In diesem System sind die zur Auswertung im Rahmen des \enquote{Learning Analytics} Aspekts herangezogenen Daten anonymisiert und es ist keine Rückverfolgung zu Individuen möglich. Bei weiteren Ausbaustufen dieser Software gilt es den Erlaubnistatbestand zu beachten. 

Neben den datenschutzrechtlichen Aspekten existieren ebenfalls ethische Bedenken, die bei der Auswertung von personenbezogenen Daten zu beachten sind. Auf folgende Punkte ist hierbei einzugehen \autocite[S.1510]{learningAnalyticsEthicalUssuesAndDilemmas}:

	Interpretierung der Daten; Zustimmung der Betroffenen; Privatsphäre und Anonymisierung der Daten; Klassifizierung und Management der Daten.
	
Diese Aspekte sind bei der Konzeption des Systems beachtet, jedoch nicht weiter ausgeführt. An dieser Stelle ist lediglich zu verstehen, dass die Erhebung der Daten transparent für die studierende Person stattfinden muss. 
	

Folgende \ac{LMS} Lehr-/ Unterrichtsaufgaben werden im Rahmen dieses Prototypensystems behandelt \autocite[S.248]{SCHOONENBOOM2014247}:
\begin{itemize}
	\item Referenzen: Dozierenden soll die Möglichkeit gegeben werden, selbst spezielle Referenzen zu ihren Vorlesungen den Kursteilnehmenden zur Verfügung zu stellen. Diese speziellen Referenzen sollen in Form von Karteikarteninhalten zur Verfügung gestellt werden, die per Definition als Werkzeug dienen.
	\item Selbsttest: Im Rahmen des Selbsttests können Studierende auf eigene Karteikarten oder auf die von Dozierenden bereitgestellten Karteikarten zurückgreifen. Hierbei kann mit den Karteikarten effizient für anstehenden Klausuren gelernt werden, was eine Lernaktivität darstellt. Als Werkzeug dient hier analog das Lernen mit Karteikarten.
	\item Außerhalb der in dieser Quelle definierten Unterrichtsaufgaben ist eine integrierte Organisationslösung für Studierende implementiert, damit Lernaktivitäten geplant und dadurch negative Ausflüsse aus einer mangelhaften Organisation minimiert werden können.	Als Werkzeuge dienen hierbei ein Kalender, der mit Funktionalitäten rund um die Klausurplanung erweitert ist. Eine direkte Lernaktivität ist mit dieser Unterrichtsaufgabe nicht realisiert, jedoch sind alle Lernaktivitäten unterstützt. Somit könnte hierbei von einer indirekten Lernaktivität gesprochen werden.
\end{itemize}

Da die Bereitstellung von Referenzen bereits mithilfe von Moodle für Dozierende möglich ist, liegt der Fokus hier explizit auf den Merhwert durch die Bereitstellung von direkt verwendbaren Karteikarten für Studierende.

Soviel zu dem grundlegenden fachlichen Kontext, im folgenden Kapitel sind diese Konzepte nochmals aufgegriffen und in konkreten Handlungsschritten umgesetzt. 

%Soll in diesem Projekt \enquote{Self-Test} eines LMSs Funktionalität umsetzen, potentiell noch weitere Funktionalitäten (Student Questions, References, Student Discussion, Blog) in dem Ansatz möglich 
%
%Tool -> instructional task -> learning activity ; für unseren Fall ausführen
%Zu \autocite[S.247]{SCHOONENBOOM2014247} Koexistenz mit Moodle darstellen und anführen, dass "wichtige" Funktionalitäten schon Moodle abgedenkt sind und eine Symbiose angestrebt ist



