% !TEX root =  ../master.tex
\chapter{Grundlage} % TODO: Kein Plan wo das hingehört
\section{Theoretische Grundlagen}
\subsection{Learning Analytics}
\subsection{Firebase-Authentisierung}


\section{Technische Grundlagen}

\subsection{Backend as a Service}
\ac{BaaS} ist ein relativ neuer Ansatz, mit dem die Entwicklung von Web- und Mobilanwendungen beschleunigt werden soll.
Bei \ac{BaaS} handelt es sich um ein Cloud-Service-Modell, welches sich aus der steigenden Nutzung von Cloud-Computing und Serverless-Ansätzen ableitet.
Dabei geht \ac{BaaS} einen Schritt weiter als andere Cloud-Ansätze wie \ac{IaaS}, \ac{CaaS} oder \ac{FaaS}.
Stattdessen orientiert sich \ac{BaaS} mehr an einem \ac{SaaS}-Ansatz. % :D
So wird in \ac{BaaS} nicht nur die gesamte Infrastruktur und Laufzeitumgebung, sondern sogar die gesamte Serverlogik von einem Cloud-Provider verwaltet.
So können sich Anwendungsentwickler auf die Entwicklung der Nutzeranwendungen und um die Geschäftslogik kümmern ohne umfangreichen Boiler-Plate-Code schreiben zu müssen.
Typische Beispiele für Services, die von einem \ac{BaaS}-Anbieter übernommen werden sind die Bereitstellung von Speicherressourcen, Nutzerauthentifizierung und das Hosting einer Website.
Diese Themenbereiche sind Services, die in fast jeder Anwendung benötigt werden, wodurch ein großer Teil der Entwicklung eingespart werden kann.
Dadurch spart es Kosten in Form von Serverequipment, Know-How und Entwicklungsressourcen.
% TODO:?




\subsection{Datenstruckturen}
\subsection{Datenschutzthematik}
