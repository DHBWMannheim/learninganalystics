% !TEX root =  master.tex
\chapter{Zusammenfassung}

\section{Fazit} % TODO: Das auch dann nochmal anpassen
In dieser Arbeit wurde eine Webanwendung erstellt, die Studenten und Dozenten im Hochschulkontext unterstützt.
Dafür wurden zuerst die funktionalen und nicht-funktionalen Anforderungen festgelegt.
Anschließend wurde ein Konzept ausgearbeitet, welches anschließend in einer Webanwendung umgesetzt wurde.
% TODO:



\section{Ausblick}
Wir sehen mit der bereits entwickelten Anwendung bereits ein großes Potenzial, welches die Organisation und den Klausurerfolg steigern kann.
Dennoch ist der Funktionsumfang der Anwendung recht limitiert.
In zukünftigen Implementierungen sehen wir besonders großes Potenzial in den folgenden Bereichen:
\begin{itemize}
    \item Live-Chat\\
        Wenn Fragen während des Lernens entstehen müssen gegenwärtig weitere Kanäle verwendet werden.
        So werden beispielsweise Google-docs-Dokumente geteilt, in den Studenten Fragen eintragen können.
        Dozenten schauen in diese Dokumente und schreiben ihre Antworten zu diesen.
        Besser wäre stattdessen ein Live-Chat, in dem alle Teilnehmer und Dozenten zu einer Vorlesung direkt befragt werden können und so das Zusammenarbeit gefördert wird.
        % TODO: Da kann man ja nebulars chat modul darstellen
    \item Import\\
        Eine weitere hilfreiche Funktion wäre der Import existierender Daten.
        So wäre es praktisch existierende Informationen aus Moodle oder Google Calendar zu importieren.
        Dadurch wäre der Umstieg in die Anwendung einfacher. 
    \item Analysen\\
        Damit sich Dozenten einen besseren Überblick über den Kurs und den Lernfortschritt machen wären auch Analysen sinnvoll.
        Beispielsweise könnte die Anwendung um Umfragen oder indirekte Datenerhebungsmethoden ergänzt werden.
\end{itemize}

% TODO: Bewertungen
% TODO: Internationalisierung
