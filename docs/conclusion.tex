% !TEX root =  master.tex
\chapter{Zusammenfassung}

\section{Fazit} % TODO: Das auch dann nochmal anpassen
In dieser Arbeit wurde eine Webanwendung erstellt, die Studenten und Dozenten im Hochschulkontext unterstützt.
Dafür wurden zuerst die funktionalen und nicht-funktionalen Anforderungen festgelegt.
Anschließend wurde ein Konzept ausgearbeitet, das danach in einer Webanwendung umgesetzt wurde.
% TODO:



\section{Ausblick}
Wir sehen mit der bereits entwickelten Anwendung bereits ein großes Potenzial, welches die Organisation und den Klausurerfolg steigern kann.
Dennoch ist der Funktionsumfang der Anwendung recht limitiert.
In zukünftigen Implementierungen sehen wir besonders großes Potenzial in den folgenden Bereichen:
\begin{itemize}
    \item Live-Chat\\
        Wenn Fragen während des Lernens entstehen müssen gegenwärtig weitere Kanäle verwendet werden.
        So werden beispielsweise Google-docs-Dokumente geteilt, in den Studenten Fragen eintragen können.
        Dozenten schauen in diese Dokumente und schreiben ihre Antworten zu diesen.
        Besser wäre stattdessen ein Live-Chat, in dem alle Teilnehmer und Dozenten zu einer Vorlesung direkt befragt werden können und so das Zusammenarbeit gefördert wird.
        % TODO: Da kann man ja nebulars chat modul darstellen
    \item Import\\
        Eine weitere hilfreiche Funktion wäre der Import existierender Daten.
        So wäre es praktisch existierende Informationen aus Moodle oder Google Calendar zu importieren.
        Dadurch wäre der Umstieg in die Anwendung einfacher. 
    \item Analysen\\
        Damit sich Dozenten einen besseren Überblick über den Kurs und den Lernfortschritt machen wären weitere Analysen sinnvoll.
        Beispielsweise könnte die Anwendung um Kurs spezifische Umfragen oder indirekte Datenerhebungsmethoden ergänzt werden.
    \item erweitere Suche\\
		Für die Studierenden würde es einen deutlichen Mehrwert, wenn durch die bereits implementierte Suchfunktion nicht nur Aufgaben, Karteikarten und Dateinamen durchsucht werden könnten, sondern auch die zu den jeweiligen Kursen hochgeladenen Dokumente. Durch den Sprung an die entsprechende Stelle im Dokument könnte die Suche von Informationen um ein vielfaches verkürzt werden und die Studenten könnten im Gegensatz zur Internetrecherche auf die Vorlesungen zugeschnittene Antworten erhalten.
\end{itemize}

Das Anwendungen wie diese bereits auf höchster politischer Ebene angekommen sind zeigt unter anderem folgende Veröffentlichung im Bundesanzeiger vom 24.02.2021: \url{https://www.bmbf.de/foerderungen/bekanntmachung-3409.html}, bei der eine Richtlinie der Bund-Länder-Initiative zur Förderung der Künstlichen Intelligenz in der Hochschulbildung, die dabei auch konkret auf die Unterstützung von Lehraktivitäten wie die in unserer Anwendung umgesetzten Methoden aus dem Bereich Learning Analytics eingeht. 
% TODO: Bewertungen
% TODO: Internationalisierung
